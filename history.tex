%!TEX root=progress_main.tex


\subsection{A Brief History of Progress in Particle Physics: 1950s to now}	\label{history}

\subsubsection{Facts and Understanding}

There can be increases in both quality and quantity of knowledge: both i) a larger number of true predictions and facts and ii) more accurate predictions and descriptions.
For simplicity, let us focus on the observation of entities.
Around 1960, three leptons were known: the electron, the muon and `a' neutrino, which are subjected to electromagnetic and weak interactions.
By now, three kinds of charged leptons are known (the $\tau $ lepton was found in 1974) and it is clear that each one comes together with a special kind of neutrino. 
In some sense, the additional knowledge follows traditional lines, it did not invoke new concepts.\footnote{This is also reflected in the terminology: the $\tau $ lepton is a 'sequential' lepton.} 
This is rather different for strongly interacting particles.
During the 1960s a growing number of strongly interacting hadrons of different 'flavours' and spins were found.
This growing number was a significant problem for physicists that was resolved by understanding hadrons as composite objects made up of quarks (see next paragraph). 
Today hadrons are instrumental for understanding the dynamics of what is considered the primary strongly interacting particles: quarks and gluons.
In addition to these matter particles, with increasing accessible energy and data rates, heavy particles with integer spins have been made that were previously predicted in the SM.

Thus, we find that progress in the accumulation of facts can be quantified only in part, while in addition, the new conceptual framework of the SM changed the direction of what facts to measure. 
Of course, within the SM, factual progress can be quantified again, for example by counting the number of observed quarks, the number of gauge bosons, etc. 
This indicates a change that itself is progress, viz., iii) the development or discovery of new avenues of research. 
In consequence, we cannot also simply talk of a smooth progress in general, but we can identify steps.
This step is intimately related to the change in conceptual framework, which we will discuss in more detail now.

Much of the conceptual progress in particle physics during the past 60 years can be seen as iv) increasing understanding.
This can be seen by looking at how physicists have increased our understanding of the interactions of particles, in particular via unification. 
This is distinct from increasing facts and predictions, because those might remain the same while we increase our understanding of how the fundamental forces fit together. 
Since gravitation is way too weak to play a role for the current energy ranges and the measurement precision, three interactions: strong, weak and electromagnetic are relevant for particle physics.

In this vein, let us consider the major concepts in the 1960s with those of today.
In the 1960s electromagnetic, weak and strong interactions appeared very different: only the electromagnetic one had an infinite reach in space, 
the strong and the weak one were only visible within some 10$^{-15}$ m and
their strengths differed by some five orders of magnitude. 
Moreover, while quantum electrodynamics was in a physical sense consistent, renormalizable and based on local gauge invariance, no such consistent theory existed for the weak and strong interactions.
The weak interaction was described by an (effective) pointlike interaction of fermions, as conceived by Fermi in 1933, and was known to only have a limited range in energy.
Within this range, fairly accurate calculations could be made, which agreed with measurements.
The strong interaction between hadrons was interpreted as an exchange of hadrons themselves (pions) with only marginal precision.
Although the concepts of quantum electrodynamics and its results were highly acknowledged, the structures of strong and weak interactions were so distinct also from QED itself, that QED was not considered a realistic blue print.
Quantum Field Theory became unpopular and replaced by other theoretical approaches like S-Matrix theory based on unitarity, analyticity and relativity, or statistical theories.
Attempts were made to unify interactions, but did turn out to be unsuccessful.

In addition, we also find that there are cases where progress along these lines has been made, but the contribution is not as direct. 
In some cases, one makes progress by v) gaining knowledge that something does not work or is not correct.
Somewhat counter-intuitively, discovering dead-end paths can also be progress. 
This can be both experimental and theoretical: one can have a prediction that fails when tested, or one may have proved a no-go theorem. 
Typically, pursuing a model or theory that turns out to be false is not characterised as progress, but we find that it is in fact progress. 
First, by closing off the number of available avenues that can be pursued, it contributes the advancement of science. 
Second, one may learn a number of things by pursuing these avenues, including, e.g., experimental know-how. 

\subsubsection{Dealing with Problems}

One can also vi) increase the number of solved problems, or vii) dissolve or redefine problems. 
By the latter, we mean that something that was formerly seen as a problem may have become a non-problem, even though it is not really solved. 
The past decades of particle physics also allows insight in how to gauge progress in terms of problem solving.
First, a simple scheme to just count the number of problems solved seems problematic since there were numerous problems in the 60s (as there are today).
Counting all of these would prevent a clear view of what has been achieved. 
As the summary of Van Hove shows, some problems are considered important, while many others that he even did not mention were assigned lesser importance.
Problems have different importance or weights, but weighting problems is not straightforward. 
Even, as is in general the case in particle physics, the field agrees on the most important problems, with time, these weights shift.
While some of the early problems are solved by today, and are still considered important, like the relation of electrons and muons and their neutrinos (now with a third $\tau $ lepton family). 
Some problems still exist but do not count as fundamental or urgent like the low $Q^2$ scattering of hadrons.
Some of the issues we now realize as fundamental problems like the existence of two states of $K^0$, now connected to CP violation, have not been realized as urgent in the 1960s.
Some problems only occur along the way of solving others or detecting new entities and processes. 
Confinement was not a problem at all in the early 1960s, but turned out to become fundamental after quarks were postulated, but experiments failed to observe free quarks.
All these examples show on the one hand that a precise measure of problem solving is debatable,  but they show clearly that it is a motor of progress---together with unexpected discoveries.

Similarly, one also make progress if viii) new problems are discovered. 
On the other hand these problem definitions should follow certain conditions.
Take interactions: the existence of three interactions was known in the 1960s and, as mentioned, consistency and unification was a goal to strive for.
Early attempts failed not the least because the experimental information was not sufficient: fundamental matter particles were considered significantly different, like hadrons being spatially extended against the point-like leptons furthermore the phenomenologies of the interactions were significantly different---one might just think of the perceived couplings that were dimensionless in case of electrodynamics, dimensionful for weak interactions.
At this stage, unification of interactions appeared to be a mere speculation.
Only after the discovery of quarks, which appeared as point-like as the fermions could all interactions be based on the same principle of local gauge theory.
Only then the interactions became comparable, all couplings became dimensionless, physicists became aware how similar the were and how they evolved with energy. 
It seems that only now unification of interactions can be meaningfully approached in the sense of defining the problem as 'how and where' do the couplings of the interactions become identical? Admitted: this problem is not solved yet and as such even today unification may be far off, but the point is that the problem itself can be put much more precisely than 60 years ago and this alone is progress.

%Maybe something about van Hove?
something

One may also quantify this kind of progress over the past decades.
The number of free parameters have been significantly reduced by the SM.
In fact, it would have been difficult to identify a complete list of fundamental free parameters by 1960, today we can count 27.
Furthermore, the theoretical progress is in part also quantifiable in terms of the precision of theoretical predictions.
The SM can calculate with astounding precision processes that are today considered fundamental, including some of the 'old' processes.  
But as discussed, for some the improvements are marginal, but they are not considered fundamental anymore and, as such, less important.