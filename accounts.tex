%!TEX root=progress_main.tex

\section{Scientific Progress} 	\label{accounts}

\subsection{A Variety of Notions}
If asked about the past sixty years of particle physics, probably every physicist would marvel at what she would call enormous progress: progress in instrumentation and measurement abilities, progress in the accumulation of facts and knowledge, and progress in theoretical understanding. 
Of course, these aspects of progress are strongly interrelated. 
The higher energies of accelerators, the significantly better measurement
precision and the capability to process more and more data increased our knowledge.
This knowledge on the other hand led, together with improved theoretical techniques, to much better understanding of the physics.
In the following, we briefly discuss the major lines of progress in particle physics over the past 60 years in order distil out the various avenues along which science can progress.
Traditional monistic concepts of progress, such as defining progress as the increase in verisimilitude or in problem-solving abilities, capture only some aspects of progress.  
In this section, we will briefly review the main accounts of progress and ultimately argue for a more pluralistic notion based on our brief historical review. 

The literature on scientific progress has attempted to answer questions about how to identify and characterise progress as well as how to measure it. 
Philosophical views on scientific progress have typically characterized it either as an increase in the truth, or truthlikeness, of scientific theories or as an increase in knowledge---either as semantic or epistemic respectively. 
Semantic notions have been developed by Popper and more recently by \citet{Niiniluoto1980-NIISP,Niiniluoto1998-NIIVTT,Niiniluoto2014-NIISPA-2} and \citet{Rowbottom2015-ROWSPW}. 
The basic idea is that a theory change can be characterised as scientific progress if the move to the new theory is closer to the truth. 
Various ways of determining this, such as the new theory having more true propositions, fewer false propositions, or more exact propositions. 
It is very convincing that if one moves from a given theory $A$ to another theory $B$ and $B$ is closer to the truth, then one has made progress. 
However, identifying that this has taken place and providing a measure on the truthlikeness of theories has proven more problematic.

These approaches have been criticized by some, such as \citet{Bird2007}, who argue that the mere accumulation of true propositions is not enough for scientific progress, since progress would be made by adding unjustified and yet true propositions to a theory. 
Bird argues instead for a notion of progress that takes into account justification as well as truth, specifically a notion based on increase in knowledge. 
This view has been criticized in that it brings with it all the problems of the truthlikeness account and more. 
This worry stems from the factive nature of knowledge---one cannot know something if it is not true. 
Thus determining if something is knowledge requires determining if it is true and so suffers the same problems as a semantic account, as well as any other issues about knowledge, such as justification, etc. 
Once again, it is rather convincing that accumulating knowledge is an sign of progress, but it is another matter altogether to measure and compare increases in knowledge and to distinguish genuine knowledge from belief.
Some accounts, such as that presented by \citet{Dellsen2016-DELSPK}, offer noetic pictures of scientific progress, where the strict requirements on justification, truth, and belief are replaced with understanding-based notions of grasping and explaining phenomena. 
According to Dellsen, science may add propositions that are true or that solve problems, but this does not constitute progress unless there is also an increase in understanding. 

Absent a view of science that linearly accumulates true propositions, justified, understood, or otherwise, a different notion of progress is required. 
Kuhn developed a historicist picture of science wherein a direct comparison of two scientific paradigms was not possible and so one could not get a clear measure of which theory is closer to the truth. 
His notion of progress is one of problem solving---a notion taken up again and developed by \citet{Laudan1981}. 
Here, a theory's ability to solve or dissolve problems gives a measure of progress.
What counts as a problem, which problems are more important, what counts as a solution, etc., are all relevant to determining progress and these are determined by the relevant scientific communities.

\citet{Mizrahi2013-MIZWIS} has taken up the novel strategy of looking at cases of real-world progress and then describing in what this progress consists.
He finds that progress can be methodological or practical as well as empirical and theoretical. 
In the end, his account is still factive and based on knowledge, but it acknowledges that progress can be made increasing \textit{knowledge how} as well as \textit{knowledge that}. 
We find that this is reasonable addition, but it is still too limited to capture the progress that we identify in the recent history of particle physics. 

The major flaw of extant accounts of progress is not that they do not capture progress, but rather their monistic nature---their commitment to reduction of all progress to a single measure.
We find that a historical examination of actual progress reveals a much richer picture of the various advances progress may consist in. 
Further, notions of progress are also focused on theory change and are blind to the smaller, yet cumulatively significant, improvements that constitute real world progress. 
Much of the literature on scientific progress concerns the comparison of two major theories in such a way that one could establish that some theory changes are not mere changes, but improvements. 
However, science also progresses within theories/paradigms/research traditions, and thus we should distinguish between \textit{local} and \textit{global} notions of progress. 
In our discussion, we will focus on exposing the multitude of sources of local progress. 

