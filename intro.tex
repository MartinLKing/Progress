%!TEX root=progress_main.tex

\section{Introduction}

Accounts of scientific progress typically attempt to capture which changes in scientific theories are advancements. 
This is often done by comparing some relevant property of two theories, such as the closeness of the theories to the truth, or the number of problems they solve. 
This is essentially a reduction of the various properties of a theory to one that is claimed to track progress. 
This reduction to one dimension is not warranted and oversimplifies the myriad changes and developments that should be seen as progressive. 
This paper will develop a pluralist notion of progress by taking a close look at the recent history of a field that has made astonishing progress in a variety of ways over the last 70-odd years, namely, particle physics. 

The argument will proceed by distinguishing global comparisons of progress between scientific theories, or across scientific theory change, from local progressive developments that may or may not lead to larger changes. 
Focusing on this latter kind of progress will reveal certain kinds of progress that have been overlooked in the literature, such as increases in experimental abilities, and taking these into account leads one to support a pluralist notion. 

In the following section (\ref{accounts}), we review the current literature on scientific progress and highlight the lack of convincing argument for the reduction of progress to one dimension. 
Further, none of the accounts is free of exceptions or problems. 
The pluralist account that will developed in the paper better reflects scientific practice and, since it accepts as progress many different kinds of changes, it has far fewer exceptions that any extant account. 
In Section~\ref{history}, we review the recent history of particle physics that demonstrates progress in many different kinds of changes and developments, including those that are experimental, rather than only theoretical. 
Finally, in Section~\ref{norm}, we will distil the pluralist account of progress and argue that not only is it more descriptively accurate, it is a better account of what we should consider as progress. 